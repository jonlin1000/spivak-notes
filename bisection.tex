\documentclass[12pt]{article}
\usepackage[margin=1in]{geometry}
\usepackage{amsmath}
\usepackage{amsthm}
\usepackage{amssymb}

\theoremstyle{definition}
\newtheorem{defn}{Definition}
\newtheorem{ex}{Example}
\newtheorem{axm}{Axiom}

\theoremstyle{plain}
\newtheorem{thm}{Theorem}
\newtheorem{prop}{Proposition}

\title{The Method of Bisection}
\date{\today}
\author{Jonathan Lin}

\begin{document}
\maketitle
\begin{abstract}
I summarize a method called the ``Bisection Argument'', which can be used to prove several important results in real analysis.
\end{abstract}

\section{The Nested Interval Theorem}
All these results rely on the following theorem:
\begin{thm}
	Suppose that $I = [a, b]$ is an interval. If we have a series of intervals $I_n$ such that $I_0 = [a, b]$ and $I_n = [a_n, b_n]$ where we have the chain
	\[I_0 \supset I_1 \supset I_2 \supset \cdots.\]

	Suppose also that 
	\[\lim_{n \to \infty}a_n - b_n = 0.\]
	Then there is a unique point $c$ contained in the $I_n$.
\end{thm}

If we let go of the requirement that $\lim_{n \to \infty} a_n - b_n = 0$, then the point $c$ still exists, but it's not necessarily unique (in fact, there will be uncountably many $c$.

\begin{proof}
The sequence $\{a_n\}$ is a monotonically increasing sequence which is bounded above by any $b_n$. Similarly, the sequence $\{b_n\}$ is a monotonically decreasing sequence which is bounded below by any $a_n$. It follows both limits exist, and we have
\[\lim_{n \to \infty} a_n = \sup(A)\] and \[\lim_{n \to \infty}b_n = \inf(B).\] Since both limits are equal (by hypothesis) we have that their limit is the unique point contained in all the $I_n$ (follows from consequences of the least upper bound).
\end{proof}

\section{Bisection of an Interval}

Here is the approach: Suppose we want to establish the existence of a point in a closed interval $[a, b]$ with property $P$. A priori we know this proposition to be true. We would like to ``seek out'' the point in question. What we might do is divide our interval in half. Then we might be able to deduce that in all cases the property $P$ holds for either $\left[a, \frac{a + b}{2}\right]$ or $\left[\frac{a + b}{2}, b\right]$. If we iteratively do this, we end up with intervals which are contained in each other with length $2^{-n}(b-a)$. After this we will be able to apply the nested interval theorem to this sequence of intervals. We summarize these above remarks by defining a \textbf{bisection} below.

\begin{defn}
	Let $I = [a, b]$ and let $P$ be some property which holds for $I$, and also has the property that it holds for $\left[a, \frac{a + b}{2}\right]$ or $\left[\frac{a + b}{2}, b\right]$. We define $I_0 = 0$ and define a sequence $I_n$ inductively as follows: given $I_n = [a_n, b_n]$ for which the property $P$ holds, then $P$ holds for either $\left[a, \frac{a_n + b_n}{2}\right]$ or $\left[\frac{a_n + b_n}{2}, b\right]$ in which case we define $I_{n + 1}$ to be the interval for which the property $P$ holds. We call this sequence a \textbf{bisection} of $I$.
\end{defn}

Note that any bisection of $I$ has the following properties: We have the chain
\[I = I_0 \supset I_1 \supset \cdots \supset I_n \supset \cdots\]
and if $I_n = [a_n, b_n]$ we have that $b_n - a_n = (b-a)2^{-n}$, so we have that 
\[\lim_{n \to \infty}b_n - a_n = 0.\] It follows that the nested interval theorem can be applied to the bisection of $I$ to get a point $c$ contained in all the $I_n$.

\section{The Intermediate Value Theorem}

\begin{thm}
Suppose $f$ is continuous on an interval $[a, b]$. Also, suppose that $f(a) \leq 0 \leq f(b)$. It follows that there exists a point $c$ in $[a, b]$ such that $f(c) = 0$.
\end{thm}

\begin{proof}
Let $P$ be the property on an interval $[a, b]$ such that $f$ is continuous on that interval and $f(a) \leq 0 \leq f(b)$. The point $(a + b)/2$ is either non-negative or non-positive. This means that $P$ holds on one of the half-intervals of $[a,b]$. Inductively, this holds as well. Hence we can construct a bisection $\{I_n\}$ where $P$ holds all the $I_n$. By the nested interval theorem there is a unique point $c$ contained in the $I_n$.

We claim that $f(c) = 0$. We can see this as follows: Since $c$ is contained in the $I_n$ we have that $f(a_n) \leq 0$ for all $a_n$ and $f(b_n) \geq 0$ for all $b_n$. Since $f$ is continuous, we have that $f(a_n) \to f(c)$ and $f(b_n) \to f(c)$. Combining these facts, we have that $f(c) \leq 0$ and $f(c) \geq 0$, implying that $f(c) = 0$.
\end{proof}

\section{The Intermediate Value Theorem}

\begin{thm}
If $f$ is continuous on $[a, b]$ then $f$ is bounded above on $[a, b]$. Moreover, $f$ attains its supremum, so that $f$ has a maximum value.
\end{thm}

\begin{proof}
Suppose that $f$ is unbounded on $[a, b]$. Then $f$ is unbounded on at least one of its half-intervals. We define the bisection $\{I_n\}$ with the property that $I_n$ is unbounded. By the nexted interval theorem, we have that 
\[\bigcap_{i = 1}^{\infty}I_n = \{c\}\] for $c \in [a, b]$. But since $f$ is continuous at $c$, there exists a $\delta > 0$ such that $|x - c| < \delta \implies |f(x) - f(c)| < 1$. So $f$ is bounded on $(c - \delta, c + \delta)$. HOwever, there exists $N \in \mathbb{N}$ such that $I_N \subset (c - \delta, c + \delta)$, which is a contradiction because $I_N$ is an unbounded set.

Clearly then we have that $S = \{f(x) \mid x \in [a,b]\}$ is bounded and non-empty. Hence $\alpha = \sup(S)$ exists. Clearly either one of the half-intervals of $[a, b]$ must have supremum $\alpha$ as well. We define the bisection $\{I_n\}$ with the property that $sup(f(I_n)) = \alpha$. By the nested interval theorem, we have that there is a unique common point $\ell$ in the $I_n$. We claim that $f(\ell) = \alpha$. This is because if $f(\ell) \neq \alpha$, then if $\varepsilon = \frac{1}{2}(\alpha - f(\ell))$ there would exist a $\delta > 0$ such that $|x - \ell| < \delta \implies |f(x) - f(\ell)| < \varepsilon$. However, there is an interval $I_n$ inside $(x - \delta, x + \delta)$, which is a contradiction since that would mean there is a value less than $\alpha$ which is an upper bound for $I_n$.
\end{proof}
\end{document}
